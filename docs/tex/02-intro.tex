\chapter*{Введение}
\addcontentsline{toc}{chapter}{Введение}

В настоящее время в мире остро стоит вопрос кибербезопасности. Существует много способов для проведения кибератаки, одним из которых является атака при помощи USB--устройства. При проведении такой атаки при подключении устройства к компьютеру можно запустить вредоносный программный код на выполнение, который может как удалить важные данные из системы, так и отправить их третьему лицу через интернет \cite{usbmalware}.

Для того, чтобы предотвратить кибератаку, проводимую посредством подключенного USB--устройства, следует строго отслеживать активные устройства в системе. С точки зрения пользовательского опыта, нет возможности запретить подключать новые устройства к компьютеру, так как большинство устройств ввода--вывода подключаются через USB, поэтому мониторинг активных устройств, их анализ  и последующее принятие решений являются хорошим способом для избежания кибератаки.

Чтобы обезопасить свои данные от утечки и передачи третьим лицам, при подключении недопустимого устройства можно отключать сетевые устройства системы. Таким образом исключается возможность передачи данных через интернет.

Цель работы --- разработать загружаемый модуля ядра Linux для отключения сетевого оборудования системы при подключении неизвестного USB--устройства