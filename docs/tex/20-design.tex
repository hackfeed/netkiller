\chapter{Конструкторская часть}

\section{IDEF0 последовательность преобразований}

На рисунках \ref{img:cp_idef_a} и \ref{img:cp_idef_b} представлена IDEF0 последовательность преобразований.

\img{110mm}{cp_idef_a}{Нулевой уровень преобразований}

\img{110mm}{cp_idef_b}{Первый уровень преобразований}

\section{Структура программного обеспечения}

В состав разрабатываемого программного обеспечения входит один загружаемый модуль ядра, который отслеживает подключенные USB--устройства и программно отключает сетевые устройства при наличии недоверенного устройства. Недоверенным устройством считается устройство, которое не идентифицируется в соответствии со списком допустимых устройств модуля. Список допустимых устройств задается в исходном коде модуля.

\section{Алгоритм отслеживания событий}

Для отслеживания событий подключения и отключения устройств в модуле ядра размещен соответствующий уведомитель, который будет зарегистрирован при загрузке модуля и удален при его удалении.

В листинге \ref{lst:notify} представлена функция обработки событий.

\begin{lstinputlisting}[
	caption={Обработка событий},
	label={lst:notify},
	style={c},
	linerange={168-190},
	]{/Users/sekononenko/Study/netkiller/src/netkiller.c}
\end{lstinputlisting}

Для каждого события есть отдельный обработчик.

\section{Алгоритм добавления подключаемых \\устройств в список подключенных\\ устройств}

Для хранения информации о подключенных устройствах будет использован связный список, хранящий информацию об идентификационных данных устройства.

Алгоритм модификации списка отслеживаемых устройств представлен в схеме на рисунке \ref{img:cp_algo}.

\section{Алгоритм работы обработчика событий}

На рисунке \ref{img:cp_algo} представлен алгоритм работы обработчика событий .

\img{200mm}{cp_algo}{Алгоритм обработчика событий}