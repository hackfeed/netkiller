\chapter{Технологическая часть}

\section{Выбор языка программирования}

Разработанный модуль ядра написан на языке программирования \texttt{C} \cite{c-language}. Выбор языка программирования \texttt{С} основан на том, что исходный код ядра Linux, все его модули и драйверы написаны на данном языке.

В качестве компилятора выбран \texttt{gcc} \cite{gcc}.

\section{Хранение информации об отслеживаемых устройствах}

Для хранения информации об отслеживаемых устройствах объявлена структура \texttt{int\_usb\_device}, которая хранит в себе идентификационные данные устройства (\texttt{PID, VID}) и указатель на элемент списка.

Структура \texttt{int\_usb\_device} представлена в листинге \ref{lst:iud}.

\begin{lstinputlisting}[
	caption={Структура \texttt{int\_usb\_device}},
	label={lst:iud},
	style={c},
	linerange={9-13},
	]{/Users/sekononenko/Study/netkiller/src/netkiller.c}
\end{lstinputlisting}

При подключении или удалении устройства, создается экземпляр данной структуры и помещается в список отслеживаемых устройств.

В листинге \ref{lst:adiud} представлены функции для работы со списком отслеживаемых устройств.

\begin{lstinputlisting}[
	caption={Функции для работы со списком отслеживаемых устройств},
	label={lst:adiud},
	style={c},
	linerange={81-102},
	]{/Users/sekononenko/Study/netkiller/src/netkiller.c}
\end{lstinputlisting}

\section{Идентификация устройства как доверенного}

Для проверки устройства необходимо проверить его идентификационные данные с данными доверенных устройств. В листинге \ref{lst:allow} представлены объявление списка доверенных устройств и функции для идентификации устройства.

\begin{lstinputlisting}[
	caption={Функции для идентификации устройств},
	label={lst:allow},
	style={c},
	linerange={16-18, 22-80},
	]{/Users/sekononenko/Study/netkiller/src/netkiller.c}
\end{lstinputlisting}

\section{Обработка событий подключения и отключения USB--устройства}

При подключении устройство добавляется в список отслеживаемых устройств. После этого происходит проверка на наличие среди отслеживаемых устройств недоверенных, и, в случае если такие были найдены, происходит отключение драйвера сети. Отключение происходит путем вызова программы \texttt{modprobe} через \texttt{usermode-helper API}.

В листинге \ref{lst:ins} представлен обработчик подключения USB--устройства.

\begin{lstinputlisting}[
	caption={Обработчик подключения USB--устройства},
	label={lst:ins},
	style={c},
	linerange={105-134},
	]{/Users/sekononenko/Study/netkiller/src/netkiller.c}
\end{lstinputlisting}

При отключении устройство удаляется из списка отслеживаемых устройств. После этого происходит проверка на наличие среди отслеживаемых устройств недоверенных, и, в случае если такие не были найдены, происходит включение драйвера сети. Включение также происходит путем вызова программы \texttt{modprobe} через \texttt{usermode-helper API}.

В листинге \ref{lst:del} представлен обработчик отключения USB--устройства.

\begin{lstinputlisting}[
	caption={Обработчик отключения USB--устройства},
	label={lst:del},
	style={c},
	linerange={137-166},
	]{/Users/sekononenko/Study/netkiller/src/netkiller.c}
\end{lstinputlisting}

\section{Регистрация уведомителя\\ для USB--устройств}

В листинге \ref{lst:not} представлено объявление уведомителя и его функции--обработчика.

\begin{lstinputlisting}[
	caption={Уведомитель для USB--устройств},
	label={lst:not},
	style={c},
	linerange={168-190},
	]{/Users/sekononenko/Study/netkiller/src/netkiller.c}
\end{lstinputlisting}

В листинге \ref{lst:mod} представлены регистрация и дерегистрация уведомителя при загрузке и удалении модуля ядра соответственно.

\begin{lstinputlisting}[
	caption={Регистрация и дерегистрация уведомителя},
	label={lst:not},
	style={c},
	linerange={192-205},
	]{/Users/sekononenko/Study/netkiller/src/netkiller.c}
\end{lstinputlisting}

\section{Примеры работы разработанного ПО}

На рисунках \ref{img:cp_ex_0} --- \ref{img:cp_ex_2} представлены примеры работы разработанного модуля ядра.

\img{50mm}{cp_ex_0}{Пример подключения доверенного устройства}

\img{24mm}{cp_ex_1}{Пример подключения недоверенного устройства}

\img{47mm}{cp_ex_2}{Пример отключения недоверенного устройства}

\section*{Вывод}

В данном разделе был обоснован выбор языка программирования, рассмотрены листинги реализованного программного обеспечения и приведены результаты работы ПО.
